\addcontentsline{toc}{part}{Class 1: Getting Started with Liquibase}
\part*{Class 1\\
Getting Started with Liquibase
}

\addcontentsline{toc}{section}{About}
{\setlength{\baselineskip}%
  {0.0\baselineskip}
  \section*{\flushright About\\}
  \hrulefill \par}
This class is the first class in a series. Rather than throw
everything at the student at once, this class is more of an
introduction on using the Liquibase tools and familiarizing the
student with database change management.

%% \phantomsection
\addcontentsline{toc}{section}{Exercise 1: Apply a database change
  with Liquibase}
{\setlength{\baselineskip}%
  {0.0\baselineskip}
  \section*{\flushright Exercise 1\\
  Apply a database change with Liquibase}
  \hrulefill \par}

\addcontentsline{toc}{subsection}{Description}
\subsection*{Description}
The student will use the base liquibase toolset to test and make a database
change. By doing this, students will familiarize themselves with the
tool, its parameters and configuration, and possible use cases for Liquibase.


\addcontentsline{toc}{subsection}{Goals}
\subsection*{Goals}
\begin{itemize}
  \item Learn about how to execute Liquibase.
  \item Explore different Liquibase parameters.
  \item Produce documentation with Liquibase.
  \item Test a liquibase change.
  \item Rollback a liquibase change.
  \item Learn about dry runs.
\end{itemize}

\addcontentsline{toc}{section}{Exercise 2: Apply a database change
  with Maven}
{\setlength{\baselineskip}%
  {0.0\baselineskip}
  \section*{\flushright Exercise 2\\
  Apply a database change with Maven}
  \hrulefill \par}

\addcontentsline{toc}{subsection}{Description}
\subsection*{Description}
The student will use the base liquibase toolset to test and make a database
change. By doing this, students will familiarize themselves with the
tool, its parameters and configuration, and possible use cases for Liquibase.


\addcontentsline{toc}{subsection}{Goals}
\subsection*{Goals}
\begin{itemize}
  \item Learn about how to execute Liquibase.
  \item Explore different Liquibase parameters.
  \item Produce documentation with Liquibase.
  \item Test a liquibase change.
  \item Rollback a liquibase change.
  \item Learn about dry runs.
\end{itemize}

\addcontentsline{toc}{subsection}{Stuff}
\newpage
  {\setlength{\baselineskip}%
           {0.0\baselineskip}
  \section*{Notes}
  \hrulefill \par}

